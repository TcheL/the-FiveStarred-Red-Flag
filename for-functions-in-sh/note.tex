%! TeX program = xelatex

\documentclass[UTF8]{ctexart}

\usepackage{titlesec}
\usepackage{fancyhdr}
\usepackage{amsmath}
\usepackage{nicefrac}

\title{关于 shell 脚本中函数的说明}
\author{}
\date{\today}

\titleformat{\section}{\Large\bf\flushleft}{\thesection}{1em}{}
\setlength{\headheight}{15pt}
\pagestyle{fancy}

\begin{document}

\maketitle

\section{determine-line-parameter}
输入 ($x_1, y_1, x_2, y_2$):
两点的坐标。

输出 ($k, b$):
两点所在直线的斜率和纵轴截距。

计算原理:
由直线方程的两点式
\[ \frac{y - y_2}{y_1 - y_2} = \frac{x - x_2}{x_1 - x_2} \]
得斜截式为
\[ y = \frac{y_1 - y_2}{x_1 - x_2}x + y_2 - \frac{y_1 - y_2}{x_1 - x_2}x_2 \]
所以,
\[ k = \frac{y_1 - y_2}{x_1 - x_2} \]
\[ b = y_2 - k x_2 \]

\section{find-twoline-crosspoint}
输入 ($k_1, b_1, k_2, b_2$):
两直线的斜截式参数。

输出 ($x, y$):
两直线交点的坐标。

计算原理:
由两直线方程联立
\[ \left\{
  \begin{aligned}
    & y = k_1 x + b_1 \\
    & y = k_2 x + b_2
  \end{aligned}
\right. \]
则
\[ k_1 x + b_1 = k_2 x + b_2 \]
所以,
\[ x = \frac{b_2 - b_1}{k_1 - k_2} \]
\[ y = k_1 x + b_1 \]

\section{find-starpoint-outercircle}
输入 ($x_0, y_0, d, \theta$):
五角星中心点坐标、外接圆直径及主顶点方位角。

输出 ($x_1^o, x_2^o, x_3^o, x_4^o, x_5^o,
y_1^o, y_2^o, y_3^o, y_4^o, y_5^o$):
五角星外圈上 outer point 的横坐标和纵坐标。

计算原理:
记 $\Delta\theta = \nicefrac{2 \pi}{5}$,
取 $\theta_i = \theta + (i - 1) \cdot \Delta\theta, (i = 1, 2, \ldots, 5)$,
则
\[ x_i^o = x_0 + \cos(\theta_i) \frac{d}{2} \]
\[ y_i^o = y_0 + \sin(\theta_i) \frac{d}{2} \]

\section{find-starpoint-innercircle}
输入 ($x_1^o, x_2^o, x_3^o, x_4^o, x_5^o,
y_1^o, y_2^o, y_3^o, y_4^o, y_5^o$):
五角星外圈上 outer point 的横坐标和纵坐标。

输出 ($x_1^i, x_2^i, x_3^i, x_4^i, x_5^i,
y_1^i, y_2^i, y_3^i, y_4^i, y_5^i$):
五角星内圈上 inner point 的横坐标和纵坐标。

计算原理:
记 $l_i^j$ 为第 i 个与第 j 个 outer point 所在的直线,
先由 determine-line-parameter 函数根据输入的点坐标确定出 $l_1^3$ 和 $l_2^5$ 的直线方程参数,
再由 find-twoline-crosspoint 函数根据得到的方程参数确定出两条直线的交点,
交点坐标即为第 1 个 inner point 的横、纵坐标。
依此类推,即可求出其他四个 inner point 的坐标。

\section{gmt-star}
输入 ($x_0, y_0, d, \theta$):
五角星中心点坐标、外接圆直径及主顶点方位角。

功能:
利用 GMT 软件绘制五角星图案。

原理:
先由 find-starpoint-outercircle 函数输入的五角星参数确定出 outer point 的坐标,
再由 find-starpoint-innercircle 函数根据得到的 outer point 坐标确定出 inner point 的坐标,
最后调用 GMT 软件依次逐点勾勒出五角星的轮廓并进行颜色填充。

\section{gmt-small-star}
输入 ($i_k, i_k, x_0, y_0$):
小五角星的网格索引号及大五角星的坐标。

功能:
利用 GMT 软件绘制小五角星图案。

原理:
先由网格索引号乘以网格间距求出小五角星的中心点坐标,
即
\[ x_k = i_k \cdot \Delta h \]
\[ y_k = j_k \cdot \Delta h \]
其中 $k = 1, 2, \ldots, 4$。
再由
\[ \theta = \arctan \left( \frac{y_i - y_0}{x_i - x_0} \right) + \pi \]
计算得到小五角星的顶点方位角,
最后由 gmt-star 函数根据得到的小五角星参数绘制出小五角星图案。

\end{document}
